% Table of contents formatting
\renewcommand{\contentsname}{Table des matières}
\setcounter{tocdepth}{3}
\setcounter{secnumdepth}{3}

\renewcommand*\listfigurename{Liste des figures}
\renewcommand*\listtablename{Liste des tableaux}

% Headers and page numbering
\usepackage{fancyhdr}
\pagestyle{plain}

% Fonts and typesetting
\setmainfont{TeX Gyre Pagella}
\setsansfont{Verdana}

% Set figure legends and captions to be smaller sized sans serif font
\usepackage[font={footnotesize,sf}]{caption}

\usepackage{siunitx}
\usepackage{float}

% Adjust spacing between lines to 1.5
\usepackage{setspace}
\onehalfspacing
\raggedbottom

% Set margins
\usepackage[top=1.25in,bottom=1.25in]{geometry}

% Chapter styling
\usepackage[grey]{quotchap}
\makeatletter
\renewcommand*{\chapnumfont}{%
  \usefont{T1}{\@defaultcnfont}{b}{n}\fontsize{80}{100}\selectfont% Default: 100/130
  \color{chaptergrey}%
}
\makeatother

% Set colour of links to black so that they don't show up when printed
\usepackage{hyperref}
\hypersetup{colorlinks=true, linkcolor=black}

% Tables
\usepackage{booktabs}
\usepackage{threeparttable}
\usepackage{array}
\newcolumntype{x}[1]{>{\centering\arraybackslash}m{#1}}

% Allow for long captions and float captions on opposite page of figures
\usepackage[rightFloats, CaptionBefore]{templates/packages/fltpage}

% Don't let floats cross subsections
\usepackage[section,subsection]{templates/packages/extraplaceins}

% University of Toulouse cover page package
\usepackage[emptypageafter, Ets=UT3]{templates/packages/tlsflyleaf/tlsflyleaf}

%% Titre, auteur, date, laboratoire, cotutelle
\title{Analyse and Modélisation de la Dynamique des Chromosomes durant la Mitose chez la levure à fission}
\author{Hadrien Mary}
\defencedate{11/12/2015}
\lab{Laboratoire de Biologie Cellulaire et Moléculaire du Contrôle de la Prolifération (UMR 5088)}
\docschool{École Doctorale Biologie Santé Biotechnologies}

%% Directeur(s) de thèse
\nboss{2}
\makesomeone{boss}{2}{Yannick Gachet}{}{}
\makesomeone{boss}{1}{Sylvie Tournier}{}{}
%% Referee
\nreferee{2}
\makesomeone{referee}{1}{Premier RAPPORTEUR}{}{}
\makesomeone{referee}{2}{Second RAPPORTEUR}{}{}
%% Jury
\njudge{3}
\makesomeone{judge}{1}{Premier MEMBRE}{Professeur d'Université}{Président du Jury}
\makesomeone{judge}{2}{Second MEMBRE}{Astronome Adjoint}{Membre du Jury}
\makesomeone{judge}{3}{Troisième MEMBRE}{Chargé de Recherche}{Membre du Jury}
